\capitulo{2}{Objetivos del proyecto}

El objetivo principal de este proyecto es el desarrollo de una aplicación de videoconferencia con asistencia remota, que establezca una comunicación entre un técnico en un ordenador y un usuario cualquiera que esté haciendo uso del dispositivo de realidad mixta HoloLens 2 de Microsoft.

Desde ITCL se marcaron unos objetivos mínimos del proyecto, los cuales eran completamente imprescindibles alcanzar para obtener como resultado una aplicación que cumpliese con las necesidades básicas de una vídeollamada asistida. 

A continuación, establecimos unos objetivos secundarios de menor relevancia que serán efectuados si el tiempo restante del proyecto lo permite. El propósito de estos es añadir funcionalidades al resultado final de la aplicación una vez cumplidos los objetivos principales. 

\section{Objetivos principales}
\begin{enumerate}
	\item Desarrollar en Unity \cite{unity:unity} una aplicación para HoloLens 2 y ordenador que permita realizar una videoconferencia con asistencia remota.
	\item Especializarme en el uso de la plataforma de desarrollo Unity.
	\item Conectar un ordenador y las HoloLens 2 mediante una aplicación en cada dispositivo, realizando las comunicaciones necesarias a través de un servidor en red local.
	\item A través de la conexión ya establecida compartir audio entre los dos dispositivos para poder entablar una conversación.
	\item Enviar en tiempo real un vídeo que muestre lo que se ve desde la perspectiva de la persona que lleva puesto el dispositivo de realidad mixta.
	\item A partir de ese vídeo recibido poder hacer clic en la pantalla del ordenador y que aparezcan indicadores en realidad aumentada al usuario que esté usando las HoloLens 2.
\end{enumerate}
\section{Objetivos secundarios}
\begin{enumerate}
	\item Implementar funcionalidades de una vídeollamada como: silenciar el micrófono, llamar, colgar, desactivar la webcam.
	\item Diseñar una interfaz gráfica para ambas aplicaciones.
	\item Realizar la conexión de las aplicaciones a través de un servidor accesible desde cualquier parte, no solo en red local.
	\item Implementar un sistema de reconocimiento de objetos.
	\item Enviar un documento en formato .pdf desde el ordenador para generarlo en realidad aumentada, haciendo así uso de él.
	\item Poder cargar modelos 3D en las HoloLens 2 para poder visualizarlos en realidad aumentada.
\end{enumerate}
\section{Objetivos técnicos}
\begin{enumerate}
	\item Conectar ambas aplicaciones usando un servidor en red local mediante la herramienta Node.js \cite{node:js}.
	\item Uso de repositorio GitHub \cite{git:github} junto con la herramienta ZenHub \cite{git:zenhub} para llevar a cabo la administración de tareas del proyecto.
	\item Documentar correctamente este prototipo para que sirva como manual para futuros desarrolladores que se entren en el proyecto.
\end{enumerate}
\section{Objetivos personales}
\begin{enumerate}

	\item Aprender a desarrollar aplicaciones en Unity \cite{unity:unity} para las HoloLens 2.
	\item Aprender a desarrollar aplicaciones para HoloLens 2.
	\item Adentrarme en el campo de la realidad aumentada y mixta.
	\item Utilizar todos los conocimientos adquiridos en la carrera.
	\item Acostumbrarme al uso de metodologías ágiles de software.
	\item Generación de una documentación mediante la herramienta \LaTeX.
\end{enumerate}