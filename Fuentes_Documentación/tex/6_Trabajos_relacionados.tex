\capitulo{6}{Trabajos relacionados}

En esta sección analizaremos y compararemos los diferentes trabajos y proyectos que cuenten con características similares a este TFG.

Debido a exclusividad de las HoloLens 2 escasea bastante la cantidad de aplicaciones desarrolladas para estas. La única que se asemeje a mi proyecto en cuanto a características y funcionalidad es software Dynamics 365 | Remote Assist desarrollado por Microsoft.

\section{Dynamics 365 | Remote Assist}

Dynamics 365 \cite{microsoft:dynamics} es un conjunto de aplicaciones desarrolladas por Microsoft que están enfocadas a un entorno empresarial. Uno de esos productos es Remote Assist, aplicación de comunicación de audio y vídeo diseñada para resolver problemas en tiempo real. Esta conecta a un experto desde un ordenador a un trabajador con un teléfono móvil u Hololens que se encuentre presencialmente en el fallo a solucionar.

A continuación, veremos una comparativa entre lo que ofrecen ambas aplicaciones: 

\newpage


\begin{table}[t]
\centering
\begin{tabular}{|l|c|c|}
\toprule
Características & Remote Assist & TFG\\
\midrule
Comunicación por vídeo & \cellcolor{green!25} Si & \cellcolor{green!25} Si\\
Comunicación por audio & \cellcolor{green!25} Si & \cellcolor{green!25} Si\\
Interfaz llamada & \cellcolor{green!25} Si & \cellcolor{green!25} Si\\
Asistencia remota & \cellcolor{green!25} Si & \cellcolor{green!25} Si\\
Seguimiento y manipulación interfaz & \cellcolor{red!25} No & \cellcolor{green!25} Si\\
Esconder la interfaz & \cellcolor{red!25} No & \cellcolor{green!25} Si\\
Envío de PDFs & \cellcolor{green!25} Si & \cellcolor{red!25} No\\
Alcance global & \cellcolor{green!25} Si & \cellcolor{red!25} No\\
Interfaz de contactos & \cellcolor{green!25} Si & \cellcolor{red!25} No\\
Almacenamiento de capturas & \cellcolor{green!25} Si & \cellcolor{red!25} No\\

\bottomrule
\end{tabular}
\caption{Comparativa de las características de las aplicaciones.}
\label{comparativa-aplicaciones}
\end{table}

Obviamente al comparar este prototipo con una aplicación desarrollada por una empresa de este calibre, vemos que mi proyecto es claramente inferior. Pero este resultado no es negativo, ya que se puede observar que ambas comparten una cantidad considerable de funcionalidades. Esto quiere decir que el prototipo desarrollado en este TFG establece una muy buena base para que un futuro, cuando ITCL retome este proyecto, supere fácilmente en funcionalidades la aplicación de Microsoft.
