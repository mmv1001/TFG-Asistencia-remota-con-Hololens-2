\apendice{Especificación de Requisitos}

\section{Introducción}
En esta sección documentaremos las funcionalidades y el alcance que va a tener el proyecto. Se mostrará en detalle los objetivos generales de la aplicación, los requisitos funcionales y los casos de uso.
\section{Objetivos generales}
El objetivo general es desarrollar una aplicación de escritorio para un ordenador con Windows y HoloLens 2 que permita:
\begin{enumerate}
    \item Iniciar la llamada.
    \item Compartir señal de vídeo.
    \item Compartir señal de audio.
    \item Asistencia remota.
    \item Interfaz para acceder a las funciones de la videoconferencia.
    \item Manipular la interfaz.
    \item Finalizar llamada.
\end{enumerate}

\section{Catálogo de requisitos}

\subsection{Requisitos funcionales}
\begin{itemize}
\tightlist
\item
    \textbf{RF-1:} Iniciar la videoconferencia.
    
    \begin{itemize}
    \tightlist
    \item
        \textbf{RF-1.1:} Generar un código desde las HoloLens 2.
    \item
        \textbf{RF-1.2:} Introducir código en el ordenador para iniciar la llamada.
        \begin{itemize}
        \tightlist
        \item
            \textbf{RF-1.2.1:} Botón de iniciar la llamada.
        \end{itemize}
    \end{itemize}
\item
    \textbf{RF-2:} Usar la videoconferencia.
    \begin{itemize}
    \tightlist
    \item
        \textbf{RF-2.1:} Visualizar la perspectiva de las HoloLens 2 desde el ordenador.
        \begin{itemize}
        \tightlist
        \item
            \textbf{RF-2.1.1:} Recibir señal de vídeo en el ordenador.
        \item
            \textbf{RF-2.1.2:} Recibir señal de audio en el ordenador.
        \end{itemize}
    \item
        \textbf{RF-2.2:} Interfaz de la aplicación del ordenador.
        \begin{itemize}
        \tightlist
        \item
            \textbf{RF-2.2.1:} Botón para generar indicadores.
        \item
            \textbf{RF-2.2.2:} Botón para deshacer el último indicador generado.
        \item
            \textbf{RF-2.2.3:} Botón para borrar todos los indicadores.
        \item
            \textbf{RF-2.2.4:} Botón para silenciar el micrófono.
        \item
            \textbf{RF-2.2.5:} Botón para apagar la \textit{webcam}.
        \end{itemize}
    \item
        \textbf{RF-2.3:} Comunicación con otro usuario desde las HoloLens 2.
        \begin{itemize}
        \tightlist
        \item
            \textbf{RF-2.3.1:} Recibir señal de vídeo en las HoloLens 2.
        \item
            \textbf{RF-2.3.2:} Recibir señal de audio en las HoloLens 2.
        \end{itemize}
    \item
        \textbf{RF-2.4:} Interfaz de la aplicación de las HoloLens 2.
        \begin{itemize}
        \tightlist
        \item
            \textbf{RF-2.4.1:} Botón para silenciar el micrófono.
        \item
            \textbf{RF-2.4.2:} Botón para apagar la \textit{webcam}.
        \item
            \textbf{RF-2.4.3:} Botón de ensordecer.
        \item
            \textbf{RF-2.4.4:} Botón del seguimiento radial de la interfaz.
        \item
            \textbf{RF-2.4.5:} Botón para ocultar la interfaz.
        \end{itemize}
    \item
        \textbf{RF-2.5:} Manipular la interfaz de las HoloLens 2.
    \item
        \textbf{RF-2.6:} Asistencia Remota.
        \begin{itemize}
        \tightlist
        \item
            \textbf{RF-2.6.1:} Enviar coordenadas desde el ordenador.
        \item
            \textbf{RF-2.6.2:} Recibir coordenadas en las HoloLens 2 y generar el indicador.
        \end{itemize}
    \end{itemize}
    
\item
    \textbf{RF-3:} Finalizar la videoconferencia.
    \begin{itemize}
    \tightlist
    \item
    \textbf{RF-3.1:} Botón de finalizar llamada.
    \item
    \textbf{RF-3.2:} Vuelta a la pantalla de inicio.
    \end{itemize}
\end{itemize}

\subsection{Requisitos no funcionales}
\begin{itemize}
    \item \textbf{RNF-1 Escalabilidad:} Este requisito es el más importante debido a que al ser un prototipo, este debe poder ser escalable ya que en un futuro la empresa ITCL lo usará como la base de un nuevo proyecto.
    \item \textbf{RNF-2 Rendimiento:} En este tipo de aplicaciones relacionadas con la realidad virtual y aumentada el rendimiento es bastante importante, ya que cuanto más fluida sea la aplicación mejor experiencia proporcionará al usuario.
    \item \textbf{RNF-3 Comodidad:} Este requisito está directamente relacionado con el anterior, puesto que cuanto mayor rendimiento posea la aplicación mayor comodidad ofrecerá al usuario, evitando así los posibles dolores de cabeza y mareos que son bastante frecuentes usando por primera vez estas tecnologías.
    \item \textbf{RNF-4 Mantenimiento:} La aplicación debe ser de fácil mantenimiento para poder ajustarse a su escalabilidad.
\end{itemize}


\subsection{Requisitos del cliente}  

El único requisito exigido por la empresa ITCL ha sido utilizar el software de Unity para el desarrollo del trabajo de fin de carrera. 

\section{Especificación de requisitos}

\begin{table}[ht!]
\centering
\begin{tabular}{|l|p{0.8\linewidth}|}
\hline
\multicolumn{2}{|l|}{\cellcolor[HTML]{C0C0C0}RF-1.1: Generar un código desde las HoloLens 2.}                                                           \\ \hline
Versión         & 1.0                                                                                                                                               \\ \hline
Autor           & Miguel Martín     \\ \hline
Descripción     & \begin{tabular}[c]{@{}p{\linewidth}@{}} La aplicación de las HoloLens 2 debe generar un código.\end{tabular} \\ \hline
Precondiciones  & Ninguna.  \\ \hline
Acciones        & El usuario abre la aplicación de las HoloLens 2.   \\ \hline
Postcondiciones & \begin{tabular}[c]{@{}p{\linewidth}@{}}El usuario comparte el código con el técnico que se encuentra en el ordenador.\end{tabular}                  \\ \hline
Excepciones     & \begin{tabular}[c]{@{}p{\linewidth}@{}}Ninguna.\end{tabular}                  \\ \hline
Importancia     & Alta                                                                                                                                          \\ \hline
\end{tabular}
\caption{RF-1.1:Generar un código desde las HoloLens 2.}
\end{table}

\begin{table}[ht!]
\centering
\begin{tabular}{|l|p{0.8\linewidth}|}
\hline
\multicolumn{2}{|l|}{\cellcolor[HTML]{C0C0C0}RF-1.2: Introducir código en el ordenador para iniciar la llamada.}                                                           \\ \hline
Versión         & 1.0                                                                                                                                               \\ \hline
Autor           & Miguel Martín     \\ \hline
Descripción     & \begin{tabular}[c]{@{}p{\linewidth}@{}}  El usuario debe introducir el código en la interfaz para poder iniciar la llamada.\end{tabular} \\ \hline
Precondiciones  & Recibir el código del otro usuario.  \\ \hline
Acciones        & El técnico introduce el código de las HoloLens 2.   \\ \hline
Postcondiciones & \begin{tabular}[c]{@{}p{\linewidth}@{}}Ninguna.\end{tabular}                  \\ \hline
Excepciones     & \begin{tabular}[c]{@{}p{\linewidth}@{}}Ninguna.\end{tabular}                  \\ \hline
Importancia     & Alta                                                                                                                                          \\ \hline
\end{tabular}
\caption{RF-1.2: Introducir código en el ordenador para iniciar la llamada.}
\end{table}


\begin{table}[ht!]
\centering
\begin{tabular}{|l|p{0.8\linewidth}|}
\hline
\multicolumn{2}{|l|}{\cellcolor[HTML]{C0C0C0}RF-1.2.1: Botón de iniciar la llamada.}                                                           \\ \hline
Versión         & 1.0                                                                                                                                               \\ \hline
Autor           & Miguel Martín     \\ \hline
Descripción     & \begin{tabular}[c]{@{}p{\linewidth}@{}} El técnico inicia la llamada tras pulsar el botón de la interfaz.\end{tabular} \\ \hline
Precondiciones  & Haber colocado el código en la interfaz.  \\ \hline
Acciones        & El técnico pulsa el botón.   \\ \hline
Postcondiciones & \begin{tabular}[c]{@{}p{\linewidth}@{}}Se inicia la videoconferencia entre las HoloLens 2 y el ordenador.\end{tabular}                  \\ \hline
Excepciones     & \begin{tabular}[c]{@{}p{\linewidth}@{}}Ninguna.\end{tabular}                  \\ \hline
Importancia     & Muy alta                                                                                                                                          \\ \hline
\end{tabular}
\caption{RF-1.2.1: Botón de iniciar la llamada..}
\end{table}


\begin{table}[ht!]
\centering
\begin{tabular}{|l|p{0.8\linewidth}|}
\hline
\multicolumn{2}{|l|}{\cellcolor[HTML]{C0C0C0}RF-2.1: Visualizar la perspectiva de las HoloLens 2 desde el ordenador.}                                                           \\ \hline
Versión         & 1.0                                                                                                                                               \\ \hline
Autor           & Miguel Martín     \\ \hline
Descripción     & \begin{tabular}[c]{@{}p{\linewidth}@{}} El técnico desde el ordenador debe ser capaz de ver la perspectiva de las HoloLens 2 y escuchar al usuario.\end{tabular} \\ \hline
Precondiciones  & Establecer la conexión.  \\ \hline
Acciones        & El técnico mira la pantalla y se coloca unos auriculares.   \\ \hline
Postcondiciones & \begin{tabular}[c]{@{}p{\linewidth}@{}}Ninguna.\end{tabular}                  \\ \hline
Excepciones     & \begin{tabular}[c]{@{}p{\linewidth}@{}}Ninguna.\end{tabular}                  \\ \hline
Importancia     & Muy alta                                                                                                                                          \\ \hline
\end{tabular}
\caption{RF-2.1: Visualizar la perspectiva de las HoloLens 2 desde el ordenador.}
\end{table}


\begin{table}[ht!]
\centering
\begin{tabular}{|l|p{0.8\linewidth}|}
\hline
\multicolumn{2}{|l|}{\cellcolor[HTML]{C0C0C0}RF-2.1.1: Recibir señal de vídeo en el ordenador.}                                                           \\ \hline
Versión         & 1.0                                                                                                                                               \\ \hline
Autor           & Miguel Martín     \\ \hline
Descripción     & \begin{tabular}[c]{@{}p{\linewidth}@{}} El técnico desde el ordenador debe ser capaz de ver la perspectiva de las HoloLens 2.\end{tabular} \\ \hline
Precondiciones  & Enviar el vídeo desde las HoloLens 2.  \\ \hline
Acciones        & El técnico ve la perspectiva del usuario.   \\ \hline
Postcondiciones & \begin{tabular}[c]{@{}p{\linewidth}@{}}Ninguna.\end{tabular}                  \\ \hline
Excepciones     & \begin{tabular}[c]{@{}p{\linewidth}@{}}Ninguna.\end{tabular}                  \\ \hline
Importancia     & Muy alta                                                                                                                                          \\ \hline
\end{tabular}
\caption{RF-2.1.1: Recibir señal de vídeo en el ordenador.}
\end{table}


\begin{table}[ht!]
\centering
\begin{tabular}{|l|p{0.8\linewidth}|}
\hline
\multicolumn{2}{|l|}{\cellcolor[HTML]{C0C0C0}RF-2.1.2: Recibir señal de audio en el ordenador.}                                                           \\ \hline
Versión         & 1.0                                                                                                                                               \\ \hline
Autor           & Miguel Martín     \\ \hline
Descripción     & \begin{tabular}[c]{@{}p{\linewidth}@{}} El técnico desde el ordenador debe ser capaz de escuchar al usuario con las HoloLens 2.\end{tabular} \\ \hline
Precondiciones  & Enviar el audio desde las HoloLens 2.  \\ \hline
Acciones        & El técnico escucha al usuario.   \\ \hline
Postcondiciones & \begin{tabular}[c]{@{}p{\linewidth}@{}}Ninguna.\end{tabular}                  \\ \hline
Excepciones     & \begin{tabular}[c]{@{}p{\linewidth}@{}}Ninguna.\end{tabular}                  \\ \hline
Importancia     & Muy alta                                                                                                                                          \\ \hline
\end{tabular}
\caption{RF-2.1.2: Recibir señal de audio en el ordenador.}
\end{table}


\begin{table}[ht!]
\centering
\begin{tabular}{|l|p{0.8\linewidth}|}
\hline
\multicolumn{2}{|l|}{\cellcolor[HTML]{C0C0C0}RF-2.2: Interfaz de la aplicación del ordenador.}                                                           \\ \hline
Versión         & 1.0                                                                                                                                               \\ \hline
Autor           & Miguel Martín     \\ \hline
Descripción     & \begin{tabular}[c]{@{}p{\linewidth}@{}} El técnico será capaz de utilizar todas las funcionalidades que ofrece la interfaz.\end{tabular} \\ \hline
Precondiciones  & Establecer la conexión.  \\ \hline
Acciones        & El técnico utiliza el ratón del ordenador para interactuar con la interfaz.   \\ \hline
Postcondiciones & \begin{tabular}[c]{@{}p{\linewidth}@{}}Se ejecuta la funcionalidad seleccionada.\end{tabular}                  \\ \hline
Excepciones     & \begin{tabular}[c]{@{}p{\linewidth}@{}}Ninguna.\end{tabular}                  \\ \hline
Importancia     & Alta                                                                                                                                          \\ \hline
\end{tabular}
\caption{RF-2.2: Interfaz de la aplicación del ordenador.}
\end{table}


\begin{table}[ht!]
\centering
\begin{tabular}{|l|p{0.8\linewidth}|}
\hline
\multicolumn{2}{|l|}{\cellcolor[HTML]{C0C0C0}RF-2.2.1: Botón para generar indicadores.}                                                           \\ \hline
Versión         & 1.0                                                                                                                                               \\ \hline
Autor           & Miguel Martín     \\ \hline
Descripción     & \begin{tabular}[c]{@{}p{\linewidth}@{}} El técnico interactúa con la interfaz para hacer aparecer objetos en realidad aumentada.\end{tabular} \\ \hline
Precondiciones  & Ninguna.  \\ \hline
Acciones        & El técnico pulsa el botón correspondiente.   \\ \hline
Postcondiciones & \begin{tabular}[c]{@{}p{\linewidth}@{}}Aparece un indicador en realidad aumenta en las HoloLens 2.\end{tabular}                  \\ \hline
Excepciones     & \begin{tabular}[c]{@{}p{\linewidth}@{}}Ninguna.\end{tabular}                  \\ \hline
Importancia     & Muy alta                                                                                                                                          \\ \hline
\end{tabular}
\caption{RF-2.2.1: Botón para generar indicadores.}
\end{table}


\begin{table}[ht!]
\centering
\begin{tabular}{|l|p{0.8\linewidth}|}
\hline
\multicolumn{2}{|l|}{\cellcolor[HTML]{C0C0C0}RF-2.2.2: Botón para deshacer el último indicador generado.}                                                           \\ \hline
Versión         & 1.0                                                                                                                                               \\ \hline
Autor           & Miguel Martín     \\ \hline
Descripción     & \begin{tabular}[c]{@{}p{\linewidth}@{}} El técnico interactúa con la interfaz para borrar el último indicador generado.\end{tabular} \\ \hline
Precondiciones  & Ninguna.  \\ \hline
Acciones        & El técnico pulsa el botón correspondiente.   \\ \hline
Postcondiciones & \begin{tabular}[c]{@{}p{\linewidth}@{}}Se borra el último indicador creado.\end{tabular}                  \\ \hline
Excepciones     & \begin{tabular}[c]{@{}p{\linewidth}@{}}Ninguna.\end{tabular}                  \\ \hline
Importancia     & Alta                                                                                                                                          \\ \hline
\end{tabular}
\caption{RF-2.2.2: Botón para deshacer el último indicador generado.}
\end{table}


\begin{table}[ht!]
\centering
\begin{tabular}{|l|p{0.8\linewidth}|}
\hline
\multicolumn{2}{|l|}{\cellcolor[HTML]{C0C0C0}RF-2.2.3: Botón para borrar todos los indicadores.}                                                           \\ \hline
Versión         & 1.0                                                                                                                                               \\ \hline
Autor           & Miguel Martín     \\ \hline
Descripción     & \begin{tabular}[c]{@{}p{\linewidth}@{}} El técnico interactúa con la interfaz para borrar todos los indicadores creados hasta el momento.\end{tabular} \\ \hline
Precondiciones  & Ninguna.  \\ \hline
Acciones        & El técnico pulsa el botón correspondiente.   \\ \hline
Postcondiciones & \begin{tabular}[c]{@{}p{\linewidth}@{}}Se borran todos las flechas de la escena.\end{tabular}                  \\ \hline
Excepciones     & \begin{tabular}[c]{@{}p{\linewidth}@{}}Ninguna.\end{tabular}                  \\ \hline
Importancia     & Alta                                                                                                                                          \\ \hline
\end{tabular}
\caption{RF-2.2.3: Botón para borrar todos los indicadores.}
\end{table}


\begin{table}[ht!]
\centering
\begin{tabular}{|l|p{0.8\linewidth}|}
\hline
\multicolumn{2}{|l|}{\cellcolor[HTML]{C0C0C0}RF-2.2.4: Botón para silenciar el micrófono.}                                                           \\ \hline
Versión         & 1.0                                                                                                                                               \\ \hline
Autor           & Miguel Martín     \\ \hline
Descripción     & \begin{tabular}[c]{@{}p{\linewidth}@{}} El técnico interactúa con la interfaz para apagar su micrófono y así dejar de enviar audio.\end{tabular} \\ \hline
Precondiciones  & Ninguna.  \\ \hline
Acciones        & El técnico pulsa el botón correspondiente.   \\ \hline
Postcondiciones & \begin{tabular}[c]{@{}p{\linewidth}@{}}Se apaga el micrófono.\end{tabular}                  \\ \hline
Excepciones     & \begin{tabular}[c]{@{}p{\linewidth}@{}}Ninguna.\end{tabular}                  \\ \hline
Importancia     & Alta                                                                                                                                          \\ \hline
\end{tabular}
\caption{RF-2.2.4: Botón para silenciar el micrófono.}
\end{table}


\begin{table}[ht!]
\centering
\begin{tabular}{|l|p{0.8\linewidth}|}
\hline
\multicolumn{2}{|l|}{\cellcolor[HTML]{C0C0C0}RF-2.2.5: Botón para apagar la \textit{webcam}.}                                                           \\ \hline
Versión         & 1.0                                                                                                                                               \\ \hline
Autor           & Miguel Martín     \\ \hline
Descripción     & \begin{tabular}[c]{@{}p{\linewidth}@{}} El técnico interactúa con la interfaz para apagar su \textit{webcam} y así dejar de enviar vídeo.\end{tabular} \\ \hline
Precondiciones  & Ninguna.  \\ \hline
Acciones        & El técnico pulsa el botón correspondiente.   \\ \hline
Postcondiciones & \begin{tabular}[c]{@{}p{\linewidth}@{}}Se apaga la \textit{webcam}.\end{tabular}                  \\ \hline
Excepciones     & \begin{tabular}[c]{@{}p{\linewidth}@{}}Ninguna.\end{tabular}                  \\ \hline
Importancia     & Alta                                                                                                                                          \\ \hline
\end{tabular}
\caption{RF-2.2.5: Botón para apagar la \textit{webcam}.}
\end{table}


\begin{table}[ht!]
\centering
\begin{tabular}{|l|p{0.8\linewidth}|}
\hline
\multicolumn{2}{|l|}{\cellcolor[HTML]{C0C0C0}RF-2.3: Comunicación con otro usuario desde las HoloLens 2.}                                                           \\ \hline
Versión         & 1.0                                                                                                                                               \\ \hline
Autor           & Miguel Martín     \\ \hline
Descripción     & \begin{tabular}[c]{@{}p{\linewidth}@{}} El usuario con las HoloLens 2 debe ser capaz comunicarse y ver al técnico que esta en el ordenador.\end{tabular} \\ \hline
Precondiciones  & Establecer la conexión.  \\ \hline
Acciones        & El usuario se coloca las HoloLens 2 en la cabeza.   \\ \hline
Postcondiciones & \begin{tabular}[c]{@{}p{\linewidth}@{}}El usuario escucha y ve la \textit{webcam} del técnico.\end{tabular}                  \\ \hline
Excepciones     & \begin{tabular}[c]{@{}p{\linewidth}@{}}Ninguna.\end{tabular}                  \\ \hline
Importancia     & Muy alta                                                                                                                                          \\ \hline
\end{tabular}
\caption{RF-2.3: Comunicación con otro usuario desde las HoloLens 2.}
\end{table}


\begin{table}[ht!]
\centering
\begin{tabular}{|l|p{0.8\linewidth}|}
\hline
\multicolumn{2}{|l|}{\cellcolor[HTML]{C0C0C0}RF-2.3.1: Recibir señal de vídeo en las Hololens 2.}                                                           \\ \hline
Versión         & 1.0                                                                                                                                               \\ \hline
Autor           & Miguel Martín     \\ \hline
Descripción     & \begin{tabular}[c]{@{}p{\linewidth}@{}} El usuario debe ser capaz de ver en una pantalla en realidad aumentada las \textit{webcam} del técnico.\end{tabular} \\ \hline
Precondiciones  & Ninguna.  \\ \hline
Acciones        & El usuario mira a la pantalla en realidad aumentada.   \\ \hline
Postcondiciones & \begin{tabular}[c]{@{}p{\linewidth}@{}}Ninguna.\end{tabular}                  \\ \hline
Excepciones     & \begin{tabular}[c]{@{}p{\linewidth}@{}}Ninguna.\end{tabular}                  \\ \hline
Importancia     & Alta                                                                                                                                          \\ \hline
\end{tabular}
\caption{RF-2.3.1: Recibir señal de vídeo en las Hololens 2.}
\end{table}


\begin{table}[ht!]
\centering
\begin{tabular}{|l|p{0.8\linewidth}|}
\hline
\multicolumn{2}{|l|}{\cellcolor[HTML]{C0C0C0}RF-2.3.2: Recibir señal de audio en las HoloLens 2.}                                                           \\ \hline
Versión         & 1.0                                                                                                                                               \\ \hline
Autor           & Miguel Martín     \\ \hline
Descripción     & \begin{tabular}[c]{@{}p{\linewidth}@{}} El usuario debe ser capaz de escuchar al técnico a través de las HoloLens 2.\end{tabular} \\ \hline
Precondiciones  & Ninguna.  \\ \hline
Acciones        & El usuario activa los altavoces del las HoloLens 2.   \\ \hline
Postcondiciones & \begin{tabular}[c]{@{}p{\linewidth}@{}}Ninguna.\end{tabular}                  \\ \hline
Excepciones     & \begin{tabular}[c]{@{}p{\linewidth}@{}}Ninguna.\end{tabular}                  \\ \hline
Importancia     & Muy alta                                                                                                                                          \\ \hline
\end{tabular}
\caption{RF-2.3.2: Recibir señal de audio en las HoloLens 2.}
\end{table}


\begin{table}[ht!]
\centering
\begin{tabular}{|l|p{0.8\linewidth}|}
\hline
\multicolumn{2}{|l|}{\cellcolor[HTML]{C0C0C0}RF-2.4: Interfaz de la aplicación de las HoloLens 2.}                                                           \\ \hline
Versión         & 1.0                                                                                                                                               \\ \hline
Autor           & Miguel Martín     \\ \hline
Descripción     & \begin{tabular}[c]{@{}p{\linewidth}@{}} El usuario será capaz de utilizar todas las funcionalidades que ofrece la interfaz de las HoloLens 2.\end{tabular} \\ \hline
Precondiciones  & Establecer la conexión.  \\ \hline
Acciones        & El usuario interactúa mediante gestos de manos con la interfaz.   \\ \hline
Postcondiciones & \begin{tabular}[c]{@{}p{\linewidth}@{}}Se ejecuta la funcionalidad seleccionada.\end{tabular}                  \\ \hline
Excepciones     & \begin{tabular}[c]{@{}p{\linewidth}@{}}Ninguna.\end{tabular}                  \\ \hline
Importancia     & Alta                                                                                                                                          \\ \hline
\end{tabular}
\caption{RF-2.4: Interfaz de la aplicación de las HoloLens 2.}
\end{table}


\begin{table}[ht!]
\centering
\begin{tabular}{|l|p{0.8\linewidth}|}
\hline
\multicolumn{2}{|l|}{\cellcolor[HTML]{C0C0C0}RF-2.4.1: Botón para silenciar el micrófono.}                                                           \\ \hline
Versión         & 1.0                                                                                                                                               \\ \hline
Autor           & Miguel Martín     \\ \hline
Descripción     & \begin{tabular}[c]{@{}p{\linewidth}@{}} El usuario interactúa con la interfaz virtual para apagar su micrófono y así dejar de enviar audio.\end{tabular} \\ \hline
Precondiciones  & Ninguna.  \\ \hline
Acciones        & El usuario pulsa el botón en realidad aumentada correspondiente.   \\ \hline
Postcondiciones & \begin{tabular}[c]{@{}p{\linewidth}@{}}Se apaga el micrófono.\end{tabular}                  \\ \hline
Excepciones     & \begin{tabular}[c]{@{}p{\linewidth}@{}}Ninguna.\end{tabular}                  \\ \hline
Importancia     & Alta                                                                                                                                          \\ \hline
\end{tabular}
\caption{RF-2.4.1: Botón para silenciar el micrófono.}
\end{table}


\begin{table}[ht!]
\centering
\begin{tabular}{|l|p{0.8\linewidth}|}
\hline
\multicolumn{2}{|l|}{\cellcolor[HTML]{C0C0C0}RF-2.4.2: Botón para apagar la \textit{webcam}.}                                                           \\ \hline
Versión         & 1.0                                                                                                                                               \\ \hline
Autor           & Miguel Martín     \\ \hline
Descripción     & \begin{tabular}[c]{@{}p{\linewidth}@{}} El usuario interactúa con la interfaz virtual para esconder la \textit{webcam} mostrada en la pantalla virtual.\end{tabular} \\ \hline
Precondiciones  & Ninguna.  \\ \hline
Acciones        & El usuario pulsa el botón en realidad aumentada correspondiente.   \\ \hline
Postcondiciones & \begin{tabular}[c]{@{}p{\linewidth}@{}}Se esconde la \textit{webcam}.\end{tabular}                  \\ \hline
Excepciones     & \begin{tabular}[c]{@{}p{\linewidth}@{}}Ninguna.\end{tabular}                  \\ \hline
Importancia     & Alta                                                                                                                                          \\ \hline
\end{tabular}
\caption{RF-2.4.2: Botón para apagar la \textit{webcam}.}
\end{table}


\begin{table}[ht!]
\centering
\begin{tabular}{|l|p{0.8\linewidth}|}
\hline
\multicolumn{2}{|l|}{\cellcolor[HTML]{C0C0C0}RF-2.4.3: Botón de ensordecer.}                                                           \\ \hline
Versión         & 1.0                                                                                                                                               \\ \hline
Autor           & Miguel Martín     \\ \hline
Descripción     & \begin{tabular}[c]{@{}p{\linewidth}@{}} El usuario interactúa con la interfaz virtual eliminar todo el sonido proveniente de la videoconferencia.\end{tabular} \\ \hline
Precondiciones  & Ninguna.  \\ \hline
Acciones        & El usuario pulsa el botón en realidad aumentada correspondiente.   \\ \hline
Postcondiciones & \begin{tabular}[c]{@{}p{\linewidth}@{}}Se silencia todo el sonido de la aplicación.\end{tabular}                  \\ \hline
Excepciones     & \begin{tabular}[c]{@{}p{\linewidth}@{}}Ninguna.\end{tabular}                  \\ \hline
Importancia     & Alta                                                                                                                                          \\ \hline
\end{tabular}
\caption{RF-2.4.3: Botón de ensordecer.}
\end{table}


\begin{table}[ht!]
\centering
\begin{tabular}{|l|p{0.8\linewidth}|}
\hline
\multicolumn{2}{|l|}{\cellcolor[HTML]{C0C0C0}RF-2.4.4: Botón del seguimiento radial de la interfaz.}                                                           \\ \hline
Versión         & 1.0                                                                                                                                               \\ \hline
Autor           & Miguel Martín     \\ \hline
Descripción     & \begin{tabular}[c]{@{}p{\linewidth}@{}} El usuario interactúa con la interfaz virtual para activar o desactivar el seguimiento radial de la interfaz virtual.\end{tabular} \\ \hline
Precondiciones  & Ninguna.  \\ \hline
Acciones        & El usuario pulsa el botón en realidad aumentada correspondiente.   \\ \hline
Postcondiciones & \begin{tabular}[c]{@{}p{\linewidth}@{}}Se activa el seguimiento radial o la interfaz se queda fija.\end{tabular}                  \\ \hline
Excepciones     & \begin{tabular}[c]{@{}p{\linewidth}@{}}Ninguna.\end{tabular}                  \\ \hline
Importancia     & Alta                                                                                                                                          \\ \hline
\end{tabular}
\caption{RF-2.4.4: Botón del seguimiento radial de la interfaz.}
\end{table}


\begin{table}[ht!]
\centering
\begin{tabular}{|l|p{0.8\linewidth}|}
\hline
\multicolumn{2}{|l|}{\cellcolor[HTML]{C0C0C0}RF-2.4.5: Botón para ocultar la interfaz.}                                                           \\ \hline
Versión         & 1.0                                                                                                                                               \\ \hline
Autor           & Miguel Martín     \\ \hline
Descripción     & \begin{tabular}[c]{@{}p{\linewidth}@{}} El usuario interactúa con la interfaz virtual para esconder toda la interfaz.\end{tabular} \\ \hline
Precondiciones  & Ninguna.  \\ \hline
Acciones        & El usuario pulsa el botón en realidad aumentada correspondiente.   \\ \hline
Postcondiciones & \begin{tabular}[c]{@{}p{\linewidth}@{}}Activas o desactivas la visualización de la interfaz.\end{tabular}                  \\ \hline
Excepciones     & \begin{tabular}[c]{@{}p{\linewidth}@{}}Ninguna.\end{tabular}                  \\ \hline
Importancia     & Alta                                                                                                                                          \\ \hline
\end{tabular}
\caption{RF-2.4.5: Botón para ocultar la interfaz.}
\end{table}


\begin{table}[ht!]
\centering
\begin{tabular}{|l|p{0.8\linewidth}|}
\hline
\multicolumn{2}{|l|}{\cellcolor[HTML]{C0C0C0}RF-2.5: Manipular la interfaz de las HoloLens 2}                                                           \\ \hline
Versión         & 1.0                                                                                                                                               \\ \hline
Autor           & Miguel Martín     \\ \hline
Descripción     & \begin{tabular}[c]{@{}p{\linewidth}@{}} El usuario de ser capaz de manipular la posición, rotación y tamaño de la interfaz virtual.\end{tabular} \\ \hline
Precondiciones  & Ninguna.  \\ \hline
Acciones        & El usuario manipulará la interfaz con gestos con las manos.   \\ \hline
Postcondiciones & \begin{tabular}[c]{@{}p{\linewidth}@{}}La interfaz modificara sus características espaciales.\end{tabular}                  \\ \hline
Excepciones     & \begin{tabular}[c]{@{}p{\linewidth}@{}}Ninguna.\end{tabular}                  \\ \hline
Importancia     & Media                                                                                                                                          \\ \hline
\end{tabular}
\caption{RF-2.5: Manipular la interfaz de las HoloLens 2}
\end{table}


\begin{table}[ht!]
\centering
\begin{tabular}{|l|p{0.8\linewidth}|}
\hline
\multicolumn{2}{|l|}{\cellcolor[HTML]{C0C0C0}RF-2.6: Asistencia Remota.}                                                           \\ \hline
Versión         & 1.0                                                                                                                                               \\ \hline
Autor           & Miguel Martín     \\ \hline
Descripción     & \begin{tabular}[c]{@{}p{\linewidth}@{}} El técnico podrá dar indicaciones al usuario a través de la generación de indicadores.\end{tabular} \\ \hline
Precondiciones  & Establecer la conexión.  \\ \hline
Acciones        & El técnico hará en el monitor y el usuario verá el resultado en las HoloLens 2.   \\ \hline
Postcondiciones & \begin{tabular}[c]{@{}p{\linewidth}@{}}Ninguna.\end{tabular}                  \\ \hline
Excepciones     & \begin{tabular}[c]{@{}p{\linewidth}@{}}Ninguna.\end{tabular}                  \\ \hline
Importancia     & Muy alta                                                                                                                                          \\ \hline
\end{tabular}
\caption{RF-2.6: Asistencia Remota.}
\end{table}


\begin{table}[ht!]
\centering
\begin{tabular}{|l|p{0.8\linewidth}|}
\hline
\multicolumn{2}{|l|}{\cellcolor[HTML]{C0C0C0}RF-2.6.1: Enviar coordenadas desde el ordenador.}                                                           \\ \hline
Versión         & 1.0                                                                                                                                               \\ \hline
Autor           & Miguel Martín     \\ \hline
Descripción     & \begin{tabular}[c]{@{}p{\linewidth}@{}} Se enviaran al coordenadas al servidor del clic realizado por el técnico en su monitor.\end{tabular} \\ \hline
Precondiciones  & Ninguna.  \\ \hline
Acciones        & Hacer clic con el ratón en la pantalla del ordenador.   \\ \hline
Postcondiciones & \begin{tabular}[c]{@{}p{\linewidth}@{}}Envío de coordenadas al servidor.\end{tabular}                  \\ \hline
Excepciones     & \begin{tabular}[c]{@{}p{\linewidth}@{}}Ninguna.\end{tabular}                  \\ \hline
Importancia     & Muy alta                                                                                                                                          \\ \hline
\end{tabular}
\caption{RF-2.6.1: Enviar coordenadas desde el ordenador.}
\end{table}


\begin{table}[ht!]
\centering
\begin{tabular}{|l|p{0.8\linewidth}|}
\hline
\multicolumn{2}{|l|}{\cellcolor[HTML]{C0C0C0}RF-2.6.2:Recibir coordenadas en las HoloLens 2 y generar el indicador.}                                                           \\ \hline
Versión         & 1.0                                                                                                                                               \\ \hline
Autor           & Miguel Martín     \\ \hline
Descripción     & \begin{tabular}[c]{@{}p{\linewidth}@{}} Las HoloLens 2 recogerán las coordenadas del servidor y crearán una flecha en las coordenadas correspondientes del espacio.\end{tabular} \\ \hline
Precondiciones  & Enviar las coordenadas al servidor.  \\ \hline
Acciones        & Las HoloLens esperan al envío de la información.   \\ \hline
Postcondiciones & \begin{tabular}[c]{@{}p{\linewidth}@{}}Generar el indicador en realidad aumentada.\end{tabular}                  \\ \hline
Excepciones     & \begin{tabular}[c]{@{}p{\linewidth}@{}}Ninguna.\end{tabular}                  \\ \hline
Importancia     & Muy alta                                                                                                                                          \\ \hline
\end{tabular}
\caption{RF-2.6.2:Recibir coordenadas en las HoloLens 2 y generar el indicador.}
\end{table}


\begin{table}[ht!]
\centering
\begin{tabular}{|l|p{0.8\linewidth}|}
\hline
\multicolumn{2}{|l|}{\cellcolor[HTML]{C0C0C0}RF-3.1: Botón de finalizar llamada}                                                           \\ \hline
Versión         & 1.0                                                                                                                                               \\ \hline
Autor           & Miguel Martín     \\ \hline
Descripción     & \begin{tabular}[c]{@{}p{\linewidth}@{}} El usuario y el técnico deberán tener la capacidad de finalizar en cualquier momento la videoconferencia.\end{tabular} \\ \hline
Precondiciones  & Estar usando la videoconferencia.  \\ \hline
Acciones        & Pulsar el botón de finalizar llamada.   \\ \hline
Postcondiciones & \begin{tabular}[c]{@{}p{\linewidth}@{}}Se termina la llamada y se realizará una transición a la pantalla de inicio.\end{tabular}                  \\ \hline
Excepciones     & \begin{tabular}[c]{@{}p{\linewidth}@{}}Ninguna.\end{tabular}                  \\ \hline
Importancia     & Alta                                                                                                                                          \\ \hline
\end{tabular}
\caption{RF-3.1: Botón de finalizar llamada}
\end{table}


\begin{table}[ht!]
\centering
\begin{tabular}{|l|p{0.8\linewidth}|}
\hline
\multicolumn{2}{|l|}{\cellcolor[HTML]{C0C0C0}RF-3.2: Vuelta a la pantalla de inicio.}                                                           \\ \hline
Versión         & 1.0                                                                                                                                               \\ \hline
Autor           & Miguel Martín     \\ \hline
Descripción     & \begin{tabular}[c]{@{}p{\linewidth}@{}} Una vez terminada la videoconferencia ambas aplicaciones vuelven a la pantalla de inicio.\end{tabular} \\ \hline
Precondiciones  & Pulsar el botón de finalizar llamada.  \\ \hline
Acciones        & Las HoloLens 2 vuelven a la pantalla de inicio.   \\ \hline
Postcondiciones & \begin{tabular}[c]{@{}p{\linewidth}@{}}Ninguna.\end{tabular}                  \\ \hline
Excepciones     & \begin{tabular}[c]{@{}p{\linewidth}@{}}Ninguna.\end{tabular}                  \\ \hline
Importancia     & Alta                                                                                                                                          \\ \hline
\end{tabular}
\caption{RF-3.2: Vuelta a la pantalla de inicio.}
\end{table}

\imagen{Casos de uso.png}{Se muestra un diagrama sobre los casos de uso derivados de los requisitos funcionales.}

La ausencia del diagrama de clases se debe a la incompatibilidad de Unity y mi aplicación con este. \textit{“A pesar de que se podría crear un diagrama de clases
de toda la aplicación, lo único que se vería sería alrededor de un par de
centenares (incluyendo los plugins) de clases inconexas que heredan de
MonoBehaviour (la clase que controla el comportamiento genérico de los
componentes de Unity), por lo que no se considera necesario.”} \footnote{Rodrigo Varga. \textit{Prototipo de simulador de carretilla elevadora Documentación Técnica.} 13 de febrero de 2018. Disponible en la biblioteca de la Universidad de Burgos.}